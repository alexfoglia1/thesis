\paragraph{Abstract}
I sistemi di posizionamento ferroviari e ferrotramviari, ad oggi impiegati, fanno un largo uso di apparati installati a terra e segnali provenienti dalla linea. La loro realizzazione ha pertanto un costo e un impatto ambientale non trascurabili.\\*
In linea con le normative europee, la ricerca nel campo del posizionamento ferroviario si sta focalizzando verso la realizzazione di sistemi di posizionamento autonomi, i quali non debbono utilizzare alcun apparato installato a terra.\\*
In questa Tesi si mostrano i risultati relativi alle attivit\`a di analisi condotte su di un sotto sistema di posizionamento autonomo basato sull'utilizzo di un algoritmo noto come \emph{Sensor Fusion Algorithm}. Tale algoritmo permette di integrare le misure fornite da un insieme di sensori installati a bordo treno, al fine di attutire il rumore di misura e ottenere una stima della posizione del treno sicura e affidabile.\\*
Viene descritto il sistema a livello hardware e a livello software e, partendo dalle stesse specifiche software, viene descritto l'ambiente di analisi realizzato al fine di condurre l'attivit\`a di \emph{monitoring} oggetto della Tesi.
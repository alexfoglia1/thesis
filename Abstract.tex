\paragraph{Abstract}
Il problema del posizionamento ferrotramviario \`e una questione di primaria importanza. Risolvere il problema del posizionamento significa poter determinare la posizione di un treno lungo una traccia ferrotramviaria. L'importanza di essere a conoscenza di tale informazione \`e fondamentale in quanto, per ragioni di rotta, un treno potrebbe avere necessit\`a di spostarsi su un nuovo binario una volta raggiunte determinate posizioni.\\*
Qualora si rendesse necessaria tale operazione, deve intervenire un sistema di scambio che direzioni le rotaie verso la nuova traccia. Il sistema di scambio \`e un sistema cosiddetto \emph{safety-critical}, in quanto un suo malfunzionamento potrebbe portare a conseguenze catastrofiche, come ad esempio il deragliamento del treno.\\*
I sistemi di posizionamento attualmente in uso fanno un largo uso di apparati installati a terra, i quali presentano un costo e un impatto ambientale non trascurabili. In questa Tesi viene mostrato un sistema di posizionamento alternativo, basato sull'utilizzo di un elaboratore installato bordo treno, sul quale viene eseguito un algoritmo \emph{Sensor Fusion}.\\*
Un algoritmo \emph{Sensor Fusion}, applicato al problema del posizionamento ferrotramviario, prende in ingresso misurazioni effettuate da un \emph{set} di sensori composto da un sensore inerziale, un odometro e un GPS; e fornisce la stima della posizione del treno pi\`u accurata di quella che si otterrebbe considerando i sensori in maniera mutuamente esclusiva. Questo avviene poich\`e il rumore casuale che caratterizza sia il moto del treno, che l'acquisizione delle misurazioni da parte dei sensori, rende sempre meno affidabili, con il procedere del tempo, le stime della posizione eventualmente basate unicamente su tali rilevazioni.\\*
Il sistema presentato utilizza un modulo software che implementa un Filtro di Kalman quale modello statistico di stima dello stato di un sistema dinamico rumoroso, in cui il sistema \`e il treno che si muove lungo una traccia, e lo stato \`e la progressiva chilometrica di quest'ultimo rispetto all'origine della traccia.
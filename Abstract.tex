\paragraph{Abstract}\mbox{}\\*
I sistemi di posizionamento ferroviari e ferrotramviari ad oggi impiegati, fanno un largo uso di apparati installati a terra e segnali provenienti dalla linea. La loro realizzazione ha pertanto un costo e un impatto ambientale non trascurabili. Per questo motivo, \`e necessario pianificare una migrazione verso di sistemi di posizionamento autonomi, i quali non debbono utilizzare alcun apparato installato a terra, in linea con le normative operazionali europee in ambito ferroviario e ferrotramviario.\\*
In questa Tesi si mostrano i risultati relativi a un' attivit\`a di \emph{fault injection} condotta su di un sotto sistema di posizionamento ferrotramviario autonomo, basato sull'utilizzo di un algoritmo noto come \emph{Sensor Fusion Algorithm}.\\*
SFA \`e un algoritmo che permette di integrare le misure fornite da un insieme di sensori installati a bordo treno, al fine di attutire il rumore di misura e ottenere una stima della posizione del treno pi\`u affidabile di quella che si otterrebbe considerando i singoli sensori.\\*
Si inquadra il sistema nel contesto dei \emph{Cyber Physical Systems of Systems} (CPSoS) e dei sistemi \emph{safety-critical}. Segue una descrizione del sistema e dell'ambiente in cui questo verr\`a analizzato; infine si mostrano e discutono i risultati sperimentali osservati durante la campagna di \emph{fault injection}.


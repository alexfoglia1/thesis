\paragraph{Abstract}\mbox{}\\*
I sistemi di posizionamento ferroviari e ferrotramviari ad oggi impiegati, fanno un largo uso di apparati installati a terra e segnali provenienti dalla linea. La loro realizzazione ha pertanto un costo e un impatto ambientale non trascurabili. Per questo motivo, \`e necessario pianificare una migrazione verso di sistemi di posizionamento autonomi, in accordo alle normative operazionali europee in ambito ferroviario e ferrotramviario definite dallo standard \texttt{ERTMS/ETCS}.\\*
Un sistema di posizionamento ferroviario, o ferrotramviario, \`e autonomo quando non fa alcun uso di apparati installati a terra.\\*
In questa Tesi si mostrano, e discutono, i risultati sperimentali ottenuti attraverso un' attivit\`a di \emph{fault injection} condotta su un sottosistema di posizionamento ferrotramviario autonomo.\\*
Il sistema \emph{target} dell'analisi basa il suo funzionamento sull'utilizzo di un insieme di sensori installati a bordo treno, le cui misure campionate vengono processate da un algoritmo noto come \emph{Sensor Fusion Algorithm} (SFA).\\*
SFA \`e un algoritmo che integra le misure fornite da un insieme di sensori al fine di attutirne il rumore di misura. L'output prodotto da SFA \`e una misura pi\`u sicura e affidabile di quella che si otterrebbe considerando i sensori singolarmente. In questo contesto, la misura che si intende fornire attraverso l'uso di SFA \`e la posizione del treno.\\*
Per le sue caratteristiche architetturali, \`e possibile classificare il sistema come un \emph{Cyber Physical Systems of Systems} (CPSoS), mentre il particolare dominio applicativo colloca il sistema nell'area \emph{safety-critical}.\\*
La Tesi passa in rassegna lo stato dell'arte circa la valutazione della \emph{dependability} di un sistema e le tradizionali tecniche di posizionamento ferroviario. Segue poi una descrizione del sistema, del suo contesto operativo nominale e degli standard che lo regolamentano. Si descrive l'ambiente in cui il sistema verr\`a analizzato, e infine si discute l'analisi condotta.

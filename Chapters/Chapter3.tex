\chapter{Architettura Software}
In questo capitolo viene completata l'esposizione dell'architettura di sistema descrivendo i moduli software in esecuzione sulla piattaforma \texttt{NVidia TX-Jetson}. I sistemi software che vengono eseguiti su detta piattaforma costituiscono l'architettura software dell'intero sottosistema di posizionamento. Detta architettura dovr\`a essere in seguito riprodotta nell'ambiente di analisi, il quale verr\`a descritto nel prossimo capitolo.
\section{Sensor Fusion Library}
Il software che esegue SFA viene fornito come libreria, denominata \textit{SensorFusionLib}.\\*
Detta libreria viene inclusa come regolare dipendenza dal software in esecuzione sulla scheda, il quale utilizza SFA sfruttandone le relative \texttt{API}.
\subsection{API}
Le \texttt{API} di \emph{SensorFusionLib} definiscono le interfacce software verso il modulo SFA che possono essere utilizzate dai \emph{client}.\\*
Le principali funzioni disponibili sono le seguenti:
\begin{itemize}
	\item \texttt{FusionInit(args...)}\\*
	Funzione che inizializza SFA. Tale funzione deve essere chiamata quando \`e necessario avviare l'algoritmo. Essa riceve come parametri i valori che caratterizzano le condizioni iniziali del moto, come progressiva chilometrica iniziale e la velocit\`a iniziale lungo i 3 assi cartesiani;
	\item \texttt{ProcessInertialMeasurementData(args...)}\\*
	Funzione che permette a SFA di ricevere ed elaborare un campionamento di IMU; 
	\item \texttt{ProcessOdometryMeasurementData(args...)}\\*
	Funzione che permette a SFA di ricevere ed elaborare un campionamento di Odometro;
	\item \texttt{ProcessGPSMeasurementData(args...)}\\*
	Funzione che permette a SFA di ricevere ed elaborare un campionamento di GPS;
	\item \texttt{ProcessStrobe(args...)}\\*
	Funzione che deve essere invocata ogni secondo, per permettere a SFA di sincronizzarsi rispetto a una \emph{global timebase} esterna; \cite{clock}
	\item 
	\texttt{IsUpdated()}\\*
	Funzione che restituisce \texttt{vero} se SFA ha completato un'iterazione ed \`e pronto a fornire l'output prodotto;
	\item \texttt{GetFusionOutput()}\\*
	Se \texttt{IsUpdated()} restituisce \texttt{vero}, \`e possibile invocare questa funzione per ricevere da SFA l'ultimo output disponibile.
\end{itemize}
\section{Interface Modules}

\chapter{Conclusioni}
Lo scopo della Tesi era quello di mostrare un sistema di posizionamento ferrotramviario basato su SFA, alternativo al sistema attualmente in uso.\\*
Nel Capitolo 1, definiti i sistemi ferroviari e i sistemi ferrotramviari, le loro similitudini e le loro differenze; \`e stato introdotto il problema del posizionamento: determinare la posizione di un treno lungo una traccia. Questo problema \`e di fondamentale importanza poich\`e il posizionamento di un treno attiva il sistema di \emph{interlocking}, il quale deve garantire un attraversamento sicuro e corretto delle JA, al fine di evitare fallimenti catastrofici.\\*
\`E stato descritto lo stato dell'arte nell'ambito del posizionamento ferrotramviario e ne sono state evidenziate le relative criticit\`a; si \`e dunque introdotto il concetto di SFA quale sistema di posizionamento ferroviario atto a risolvere molte delle criticit\`a esposte.\\*
Nel Capitolo 2 \`e stato formalizzato, da un punto di vista matematico, il concetto di sistema dinamico, quale \`e un treno in movimento lungo una traccia ferrotramviaria. In particolare, sono stati formalizzati i concetti di rumore di processo e rumore di misura. In quest'ottica, sono stati introdotti i KF come base matematica di un algoritmo in grado di stimare lo stato di un sistema dinamico caratterizzato da rumore intrinseco, attraverso misurazioni caratterizzate anch'esse da rumore..\\*
Nel Capitolo 3 \`e esposta un' architettura essenziale di un sistema di posizionamento ferrotramviario che sia in grado di sfruttare un algoritmo SFA.\\*
Sono state evidenziate le criticit\`a di tale architettura e le relative mitigazioni.\\*
Il Capitolo 4 rappresenta infine la parte \emph{sperimentale} della Tesi: sono mostrati gli esperimenti effettuati, i risultati ottenuti, e di conseguenza gli argomenti a supporto dell'utilizzo di SFA come sistema di posizionamento ferrotramviario.

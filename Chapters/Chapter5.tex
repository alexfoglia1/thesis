\chapter{Parte Sperimentale}
\definecolor{mygreen}{rgb}{0.0.33, 0.42, 0.18}
Questo capitolo rappresenta la parte sperimentale del lavoro di Tesi.\\*
L'attivit\`a di analisi effettuata rientra nella categoria del \emph{monitoring} dei sistemi software.\\*
Nel seguito si illustrano gli esperimenti condotti sul sistema nell'ambiente descritto nel capitolo 4, e si discutono i risultati ottenuti.\\*
La traccia ferrotramviaria scelta per l'analisi \`e un sottoinsieme della linea \texttt{T1} della Tramvia di Firenze, che si estende per $4.25044$ chilometri dal terminal di \emph{Villa Costanza}, nel comune di Scandicci. 
\begin{figure}[h]
	\centering
	\includegraphics[width=0.7\linewidth]{"../Trainpositioning/linea test"}
	\caption{Traccia di analisi}
	\label{fig:linea-test}
\end{figure}
\section{Misure di interesse}
La \emph{fault tolerance} del sistema verr\`a valutata in termini degli errori che SFA commette nella stima delle seguenti grandezze:
\begin{itemize}
	\item Posizione del treno, espressa in coordinate ECEF;
	\item Velocit\`a del treno, proiettata sui tre assi cartesiani.
\end{itemize}
ECEF \`e l'acronimo di \emph{Earth Centered Earth Fixed}. Una coordinata ECEF esprime la posizione di un oggetto immerso in un sistema di riferimento cartesiano a 3 dimensioni, con origine nel centro della Terra.\\*
Si iniettano i seguenti guasti:
\begin{itemize}
	\item Sopressione del canale di comunicazione tra odometro e SFA (Scenario 1);
	\item Soppressione del canale di comunicazione tra IMU e SFA (Scenario 2);
\end{itemize}
Per ciascuna grandezza osservata viene riportato il valore medio, il valore
massimo e la deviazione standard (dev. std.) dell'errore commesso da SFA nel relativo processo di stima.
\section{Esperimenti}
\subsection{Golden Run}
Per ottenere un termine di paragone consistente, il sistema viene eseguito prima senza l'iniezione di guasti.\\*
I risultati ottenuti durante la campagna di \emph{fault injection} verranno quindi valutati in relazione ai risultati ottenuti dal software senza introduzione di guasti. Questa prima esecuzione del software prende il nome di \emph{golden run}.
\begin{table}[h]
	\centering
	\begin{tabular}{|p{3.25cm}|p{2cm}|p{2cm}|p{2cm}|p{2cm}|}
		\hline 
		\textbf{Sensori integrati} & \textbf{Frequenza IMU}  & \textbf{Frequenza odometro} & \textbf{Varianza Odometro} & \textbf{Iterazioni} \\ 
		\hline 
		IMU, Odometro & 100 Hz & 10 Hz & 0.0004 & 10 \\
		\hline 
	\end{tabular}
	\caption{Golden Run: workload}
	\label{tab:exp12}
\end{table}
\begin{table}[h]
	\centering
	\begin{tabular}{|p{2cm}|p{3cm}|p{3cm}|p{3cm}|}
		\hline 
		\textbf{Misura} & \textbf{Errore medio}  & \textbf{Errore massimo} & \textbf{Dev. std. errore}\\ 
		\hline 
		ECEF X & 3.5826 m & 20.1141 m & 5.60308 m \\ 
		\hline 
		ECEF Y & 0.0243133 m & 0.362813 m & 0.0452763 m \\ 
		\hline 
		ECEF Z & 3.56432$\cdot10^{-6}$ m & 3.19201$\cdot10^{-5}$ m & 8.81543$\cdot10^{-6}$ m \\ 
		\hline 
		Velocit\`a X & 0.0169528 m/s & 0.124472 m/s & 0.0199173 m/s \\ 
		\hline 
		Velocit\`a Y & 0.0394826 m/s & 0.847261 m/s & 0.0828195 m/s \\ 
		\hline 
		Velocit\`a Z & 0.00382241 m/s & 0.0192343 m/s & 0.00314704 m/s \\ 
		\hline 
	\end{tabular}
	\caption{Golden Run: Risultati}
	\label{tab:exp12res}
\end{table}
\newpage
\subsection{Scenario 1}
In questo scenario il \emph{faultload} consiste nella soppressione del canale di comunicazione tra Odometro e modulo SFA.
\subsubsection{Esperimento 1.1}
Si sopprime il canale di comunicazione \textbf{per tutta la durata dell'esperimento}.\\*
\begin{table}[h]
	\centering
	\begin{tabular}{|p{3.25cm}|p{2cm}|p{2cm}|p{2cm}|p{2cm}|}
		\hline 
		\textbf{Sensori integrati} & \textbf{Frequenza IMU}  & \textbf{Frequenza odometro} & \textbf{Varianza Odometro} & \textbf{Iterazioni} \\ 
		\hline 
		IMU, Odometro & 100 Hz & 10 Hz & 0.0004 & 10 \\
		\hline 
	\end{tabular}
	\caption{Esperimento 1.1: workload}
\end{table}
	\begin{table}[h]
	\centering
	\begin{tabular}{|p{2cm}|p{3cm}|p{3cm}|p{3cm}|}
		\hline 
		\textbf{Misura} & \textbf{Errore medio}  & \textbf{Errore massimo} & \textbf{Dev. std. errore}\\ 
		\hline 
		ECEF X & 861.883 m & 2431.1 m & 678.953 m \\ 
		\hline 
		ECEF Y & 348.814 m & 1518.65 m & 499.222 m \\ 
		\hline 
		ECEF Z & 0.123305 m & 0.155086 m & 0.567656 m \\ 
		\hline 
		Velocit\`a X & 8.20331 m/s & 30.782 m/s & 7.32822 m/s \\ 
		\hline 
		Velocit\`a Y & 23.4213 m/s & 75.1929 m/s & 20.0333 m/s \\ 
		\hline 
		Velocit\`a Z & 87.7399 m/s & 87.1907 m/s & 245.723 m/s \\ 
		\hline 
	\end{tabular}
	\caption{Esperimento 1.1: Risultati}
	\label{tab:exp11res}
\end{table}
\begin{table}[h]
	\centering
	\begin{tabular}{|p{2cm}|p{3.2cm}|p{3cm}|p{3cm}|}
		\hline 
		\textbf{Misura} 
		& \textbf{Errore medio} 
		& \textbf{Errore massimo}
		& \textbf{Dev. std. errore}\\ 
		\hline 
		ECEF X & \textcolor{red}{\textbf{+23957.5 \%}}& \textcolor{red}{\textbf{+11986.5 \%}} & \textcolor{red}{\textbf{+12017.5 \%}}  \\ 
		\hline 
		ECEF Y & \textcolor{red}{\textbf{+1.43456e+06 \%}}& \textcolor{red}{\textbf{+4.18477e+05 \%}} & \textcolor{red}{\textbf{+1.10251e+06 \%}}  \\ 
		\hline 
		ECEF Z & \textcolor{red}{\textbf{+3.45933e+06 \%}}& \textcolor{red}{\textbf{+1.77827e+06 \%}} & \textcolor{red}{\textbf{+1.75916e+06 \%}}  \\ 
		\hline 
		Velocit\`a X & \textcolor{red}{\textbf{+48289.1 \%}}& \textcolor{red}{\textbf{+24630.1 \%}} & \textcolor{red}{\textbf{+36693.2 \%}}  \\ 
		\hline 
		Velocit\`a Y & \textcolor{red}{\textbf{+59220.6 \%}}& \textcolor{red}{\textbf{+8774.82 \%}} & \textcolor{red}{\textbf{+24089.1 \%}}  \\ 
		\hline 
		Velocit\`a Z & \textcolor{red}{\textbf{+2.29531e+06 \%}}& \textcolor{red}{\textbf{+1.27743e+06 \%}}& \textcolor{red}{\textbf{+2.77046e+06 \%}} \\ 
		\hline 
	\end{tabular} 
	\caption{Esperimento 1.1: Golden Run} 
\end{table}
\FloatBarrier
Alimentare SFA utilizzando esclusivamente le misurazioni di IMU conducono a una netta divergenza dell'errore, sia in termini di posizione che in termini di velocit\`a.\\*
Questi risultati non sono accettabili.
\subsubsection{Esperimento 1.2}
Si sopprime il canale di comunicazione tra SFA e Odometro durante \textbf{la prima met\`a della simulazione}.\\*  
\begin{table}[h]
	\centering
\begin{tabular}{|p{3.25cm}|p{2cm}|p{2cm}|p{2cm}|p{2cm}|}
	\hline 
	\textbf{Sensori integrati} & \textbf{Frequenza IMU}  & \textbf{Frequenza odometro} & \textbf{Varianza Odometro} & \textbf{Iterazioni} \\ 
	\hline 
	IMU, Odometro & 100 Hz & 10 Hz & 0.0004 & 10 \\
	\hline 
\end{tabular}
	\caption{Esperimento 1.2: workload}
\end{table}
\FloatBarrier
\begin{table}[h]
	\centering
	\begin{tabular}{|p{2cm}|p{3.2cm}|p{3cm}|p{3cm}|}
		\hline 
		\textbf{Misura} 
		& \textbf{Errore medio} 
		& \textbf{Errore massimo}
		& \textbf{Dev. std. errore}\\ 
		\hline 
		ECEF X & 4.3248 m & 20.5617 m & 5.41018 m \\ 
		\hline 
		ECEF Y & 0.0226056 m & 0.39742 m & 0.04816 m \\ 
		\hline 
		ECEF Z & 3.81041e-06 m & 3.30294e-05 m & 9.04865e-06 m \\ 
		\hline 
		Velocit\`a X & 0.0248871 m/s & 0.150687 m/s & 0.0260141 m/s \\ 
		\hline 
		Velocit\`a Y & 0.0475246 m/s & 0.908185 m/s & 0.0899591 m/s \\ 
		\hline 
		Velocit\`a Z & 0.00406042 m/s & 0.0228906 m/s & 0.00340827 m/s \\ 
		\hline 
	\end{tabular} 
	\caption{Esperimento 1.2: Risultati}
\end{table}
\begin{table}[h]
	\centering
	\begin{tabular}{|p{2cm}|p{3.2cm}|p{3cm}|p{3cm}|}
		\hline 
		\textbf{Misura} 
		& \textbf{Errore medio} 
		& \textbf{Errore massimo}
		& \textbf{Dev. std. errore}\\ 
		\hline 
		ECEF X & \textcolor{red}{\textbf{+20.7168 \%}}& \textcolor{red}{\textbf{+2.22531 \%}} & \textcolor{mygreen}{\textbf{-3.44275 \%}}  \\ 
		\hline 
		ECEF Y & \textcolor{mygreen}{\textbf{-7.02373 \%}}& \textcolor{red}{\textbf{+9.53852 \%}} & \textcolor{red}{\textbf{+6.36912 \%}}  \\ 
		\hline 
		ECEF Z & \textcolor{red}{\textbf{+6.90426 \%}}& \textcolor{red}{\textbf{+3.47524 \%}} & \textcolor{red}{\textbf{+2.64559 \%}}  \\ 
		\hline 
		Velocit\`a X & \textcolor{red}{\textbf{+46.8023 \%}}& \textcolor{red}{\textbf{+21.061 \%}} & \textcolor{red}{\textbf{+30.6106 \%}}  \\ 
		\hline 
		Velocit\`a Y & \textcolor{red}{\textbf{+20.3685 \%}}& \textcolor{red}{\textbf{+7.1907 \%}} & \textcolor{red}{\textbf{+8.62068 \%}}  \\ 
		\hline 
		Velocit\`a Z & \textcolor{red}{\textbf{+6.2267 \%}}& \textcolor{red}{\textbf{+19.0093 \%}}& \textcolor{red}{\textbf{+8.30082 \%}} \\ 
		\hline 
	\end{tabular} 
	\caption{Esperimento 1.2: Confronto con golden run} 
\end{table}
Si osserva un lieve degrado delle performance che influisce in maniera non particolarmente significativa ai fini del posizionamento.
\subsubsection{Esperimento 1.3}
Si sopprime il canale di comunicazione tra SFA e Odometro durante \textbf{la seconda met\`a della simulazione}.\\* 
\begin{table}[h]
	\centering
\begin{tabular}{|p{3.25cm}|p{2cm}|p{2cm}|p{2cm}|p{2cm}|}
	\hline 
	\textbf{Sensori integrati} & \textbf{Frequenza IMU}  & \textbf{Frequenza odometro} & \textbf{Varianza Odometro} & \textbf{Iterazioni} \\ 
	\hline 
	IMU, Odometro & 100 Hz & 10 Hz & 0.0004 & 10 \\
	\hline 
\end{tabular}
	\caption{Esperimento 1.3: workload}
\end{table}
\begin{table}[h]
	\centering
	\begin{tabular}{|p{2cm}|p{3.2cm}|p{3cm}|p{3cm}|}
	\hline 
	\textbf{Misura} 
	& \textbf{Errore medio} 
	& \textbf{Errore massimo}
	& \textbf{Dev. std. errore}\\ 
	\hline 
	ECEF X & 3.57373 m & 20.1609 m & 5.60304 m \\ 
	\hline 
	ECEF Y & 0.0234386 m & 0.366496 m & 0.0445943 m \\ 
	\hline 
	ECEF Z & 3.55578e-06 m & 3.19863e-05 m & 8.7636e-06 m \\ 
	\hline 
	Velocit\`a X & 0.0184494 m/s & 0.129497 m/s & 0.0222426 m/s \\ 
	\hline 
	Velocit\`a Y & 0.0396467 m/s & 0.863711 m/s & 0.084737 m/s \\ 
	\hline 
	Velocit\`a Z & 0.00355928 m/s & 0.0189619 m/s & 0.00306268 m/s \\ 
	\hline 
\end{tabular} 
	\caption{Esperimento 1.3: Risultati}
\end{table}
\begin{table}[h]
	\centering
	\begin{tabular}{|p{2cm}|p{3.2cm}|p{3cm}|p{3cm}|}
	\hline 
	\textbf{Misura} 
	& \textbf{Errore medio} 
	& \textbf{Errore massimo}
	& \textbf{Dev. std. errore}\\ 
	\hline 
	ECEF X & \textcolor{mygreen}{\textbf{-0.247586 \%}}& \textcolor{red}{\textbf{+0.232673 \%}} & \textcolor{mygreen}{\textbf{-0.000714 \%}}  \\ 
	\hline 
	ECEF Y & \textcolor{mygreen}{\textbf{-3.59762 \%}}& \textcolor{red}{\textbf{+1.01512 \%}} & \textcolor{mygreen}{\textbf{-1.50631 \%}}  \\ 
	\hline 
	ECEF Z & \textcolor{mygreen}{\textbf{-0.239597 \%}}& \textcolor{red}{\textbf{+0.207393 \%}} & \textcolor{mygreen}{\textbf{-0.587946 \%}}  \\ 
	\hline 
	Velocit\`a X & \textcolor{red}{\textbf{+8.82804 \%}}& \textcolor{red}{\textbf{+4.03705 \%}} & \textcolor{red}{\textbf{+11.6748 \%}}  \\ 
	\hline 
	Velocit\`a Y & \textcolor{red}{\textbf{+0.415626 \%}}& \textcolor{red}{\textbf{+1.94155 \%}} & \textcolor{red}{\textbf{+2.31528 \%}}  \\ 
	\hline 
	Velocit\`a Z & \textcolor{mygreen}{\textbf{-6.88388 \%}}& \textcolor{mygreen}{\textbf{-1.41622 \%}}& \textcolor{mygreen}{\textbf{-2.68061 \%}} \\ 
	\hline 
\end{tabular} 
	\caption{Esperimento 1.3: Confronto con golden run} 
\end{table}
\FloatBarrier
Non si osservano particolari variazioni rispetto ai risultati ottenuti nella \emph{golden run}. \`E ragionevole ipotizzare che la causa di questo comportamento sia da ricercare nella sezione di traccia in cui si \`e soppresso il canale di comunicazione tra Odometro e SFA. Nella prima met\`a la traccia compie una curva netta, mentre nella seconda essa si sviluppa in modo approssimativamente lineare.
\subsection{Scenario 2}
In questo scenario si sopprime il canale di comunicazione tra IMU e il modulo SFA.\\*
\subsubsection{Esperimento 2.1}
\subsubsection{Esperimento 2.2}
\subsubsection{Esperimento 2.3}
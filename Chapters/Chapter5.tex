\chapter{Esperimenti e Risultati}
Questo capitolo rappresenta la parte sperimentale del lavoro di Tesi.\\*
L'attivit\`a di analisi effettuata rientra nella categoria del \emph{monitoring} dei sistemi software.\\*
Nel seguito, si illustrano gli esperimenti condotti sul sistema nell'ambiente descritto nel capitolo 4, e si discutono i risultati ottenuti.
\section{Traccia di analisi}
La traccia ferrotramviaria scelta per l'analisi \`e un sottoinsieme della linea \texttt{T1} della Tramvia di Firenze.\\*
\`E stato scelto un sottoinsieme della linea, in luogo della linea intera, principalmente per ragioni di complessit\`a computazionale.
\section{Misure di interesse}
Prima di poter procedere all'analisi del sistema, occorre definire le \emph{misure di interesse} che si intende valutare attraverso gli esperimenti condotti.
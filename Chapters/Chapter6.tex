\chapter{Conclusioni}
Lo scopo della Tesi era mostrare i risultati sperimentali ottenuti durante le attivit\`a di analisi condotte su un sistema di posizionamento ferrotramviario innovativo.\\*
Al giorno d'oggi, i costruttori dei sistemi ferrotramviari tendono a rispettare le normative operazionali imposte dallo standard europeo \emph{ERTMS}.\\*
\emph{ERTMS} nasce per uniformare i regolamenti dei paesi dell'Unione Europea in materia ferroviaria. \`E necessario uniformare detti regolamenti poich\`e le linee ferroviarie non sono pi\`u esclusivamente limitate a territori nazionali.\\*
\emph{ETCS} \`e la parte di \emph{ERTMS} che regolamenta il posizionamento dei rotabili. La filosofia di \emph{ETCS} mira a realizzare sistemi di posizionamento completamente autonomi, conformi al livello \texttt{ETCS-3}, i quali non fanno alcun uso di apparati installati a terra.\\*
Nell'ottica di evoluzione verso una completa autonomia, \`e stato mostrato ed analizzato un possibile sistema di posizionamento autonomo operante in un contesto ferrotramviario. Tale sistema fa uso di un insieme di sensori installati a bordo treno, capaci di campionare le grandezze fisiche che caratterizzano il moto del medesimo: accelerazione, velocit\`a angolare, velocit\`a lineare e coordinate geografiche.\\*
Il rumore che caratterizza ciascun processo di misura viene attenuato dall'utilizzo di un algoritmo SFA, che integra le misure campionate dai sensori al fine di stimare in modo affidabile la posizione del treno lungo la traccia entro cui si sta muovendo.\\*

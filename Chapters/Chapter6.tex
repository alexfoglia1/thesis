\chapter{Conclusioni}
In questa Tesi si \`e discusso un processo di analisi sperimentale condotto verso un sottosistema di posizionamento ferrotramviario.\\*
L'analisi condotta rientra nella categoria \emph{fault injection}, la quale prevede l'osservazione diretta del sistema, o di un suo prototipo, nel suo reale ambiente di esecuzione. Il fine di un'attivit\`a di \emph{fault injection} \`e quello di raccogliere accurate misure sperimentali circa la \emph{dependability} del sistema analizzato, quando vengono inseriti volontariamente dei guasti al suo interno. Si \`e discusso il concetto di \emph{dependability}, definendola come la capacit\`a che ha un sistema di fornire un servizio in modo corretto. In particolare, sono stati individuati:
\begin{itemize}
	\item Le \emph{measures}, ovvero i parametri di valutazione della \emph{dependability};
	\item I \emph{threats}, ossia gli eventi che minano la \emph{dependability} di un sistema;
	\item I \emph{means}, insieme di tecniche e metodologie atte a raggiungere la \emph{dependability} di un sistema informatico.
\end{itemize}
La \emph{dependability} \`e un aspetto fondamentale per tutti i sistemi informatici, ma il suo raggiungimento diventa obbligatorio per sistemi operanti in contesti \emph{safety critical}, come ad esempio il settore ferroviario.\\*
Sono stati discussi gli standard \texttt{EN 50126} e \texttt{EN 50128} che regolamentano la gestione e il raggiungimento della \emph{safety} nei sistemi ferroviari, e le normative operazionali imposte dallo standard europeo \texttt{ERTMS/ETCS}, completando il quadro del contesto normativo e operativo in cui si colloca il sistema studiato.\\*
Il sistema \`e stato poi classificato da un punto di vista architetturale, come un \emph{Cyber Physical System of Systems}, quindi sono stati descritti gli elementi chiave, come le interfacce e i sistemi costituenti. La definizione delle interfacce \`e particolarmente importante poich\`e qualunque struttura del sistema, responsabile del comportamento osservato, pu\'o essere ridotta alla specifica delle interfacce del sistema.\\*
La descrizione del sistema si \`e quindi conclusa con la specifica dei software che lo compongono.\\*
Definito il sistema nominale, la discussione si \`e concentrata sulla descrizione dell'ambiente di analisi costruito al fine di garantire le propriet\`a che un sistema di \emph{monitoring} e \emph{fault injection} deve possedere:
\begin{itemize}
	\item Non intrusivit\`a;
	\item Rappresentativit\`a;
	\item Ripetibilit\`a;
	\item Fattibilit\`a.
\end{itemize}
Sono stati quindi descritti gli esperimenti condotti individuando:
\begin{itemize}
	\item Un \emph{fault model} rappresentativo per il sistema;
	\item Le misure di interesse che si intende valutare;
	\item Un \emph{workload} il pi\`u possibile conforme agli input reali che il sistema dovr\`a processare;
	\item Un \emph{faultload} basato sui requisiti di sistema e sulle tecnologie impiegate.
\end{itemize}
Attraverso l'attivit\`a di \emph{fault injection} \`e stato principalmente osservato che:
\begin{enumerate}
	\item  Il sistema \`e in grado di tollerare bene guasti al canale di comunicazione verso i sensori, a condizione che IMU sia sempre funzionante e che la traccia sia sufficientemente lineare;
	\item Il sistema \`e capace di individuare messaggi che hanno un' elevata probabilit\`a di essere errati, e quindi di correggerne il contenuto;
	\item Se IMU non fornisce messaggi al sistema per pi\`u di cinque secondi, questo termina la sua esecuzione in una maniera comunque rilevabile facilmente da altri meccanismi di controllo installati a bordo.
\end{enumerate}
I risultati che ha fornito la campagna di \emph{fault injection} sembrano promettenti nell'ottica di poter impiegare il sistema sul campo, sebbene questo sia ancora in fase di sviluppo e si \`e reso disponibile per l'analisi solamente un suo prototipo.\\*
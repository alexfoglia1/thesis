\chapter{Sensor Fusion}
Nei sistemi in cui \`e richiesta un'alta \emph{reliability} delle misure, l'informazione fornita dai singoli sensori non \`e sufficiente. In questi casi \`e raccomandato l'utilizzo di un insieme di sensori in contemporanea.
\section{Panoramica}
In generale, un algoritmo SFA viene utilizzato per stimare lo stato di un sistema dinamico in un ambiente caratterizzato da \emph{rumore}.
\subsection{Sistemi Dinamici}
Un sistema dinamico \`e una modellazione matematica di un processo che evolve nel tempo, la cui evoluzione \`e descritta attraverso un sistema di equazioni differenziali o alle differenze, nel caso esso si evolva rispettivamente a tempo continuo o a tempo discreto.\\*
Sia $S$ l'insieme dei possibili stati che il sistema pu\`o assumere, e sia $m = |S|$ la dimensione dello spazio degli stati.\\*
Senza perdere in generalit\`a, si possono formalizzare questi due tipi di sistemi dinamici come:
$$
y'(t) = f(t,y(t)),\;t\ge 0\;\;\;\;(1)
$$
Con $y(0) \in \mathbb{R}^m$ condizione iniziale nota, e:
$$
y_{n+1} = f(n,y_n),\;n = 0,1,\dots\;\;\;\;(2)
$$
con al solito $y_0 \in \mathbb{R}^m$ condizione iniziale nota.\\*
Ricavare lo stato del sistema dinamico per un certo istante $t$, o $n$, equivale a risolvere le equazioni cui sopra e valutarne la traiettoria soluzione in $t$ o in $n$.\\*
Un semplice sistema dinamico \`e rappresentato da un punto materiale che si muove con una accelerazione costante $$\mathbf{a} = a\mathbf{k}$$ dove $\mathbf{k}$ \`e un qualunque versore della base canonica di $\mathbb{R}^3$.\\*
Supponendo che il punto si muova con velocit\`a iniziale $\mathbf{z'}(0) = v_0\mathbf{k}$ nota e inizi il moto da una coordinata $\mathbf z(0) = z_0\mathbf{k}$ nota, si ha:
$$
z''(t) = a\;\;\;\;(3)
$$
$$
z'(t) = \int{a dt} = a t + v_0
$$
$$
z(t) = \int(a t + v_0)dt = \frac{1}{2} at^2 +  v_0 t + z_0
$$
L'equazione $z(t)$ descrive completamente la traiettoria di moto del punto materiale, mentre $z'(t)$ descrive completamente la traiettoria della velocit\`a del punto durante il suo moto.
\subsection{Misure e Rumore}
In questo semplice esempio, viene fatta l'assunzione di conoscere a priori il valore esatto di $a$, di $v_0$ e di $z_0$.\\*
Nella pratica, per misurare l'accelerazione $a$ \`e necessario uno strumento denominato \emph{accelerometro}, il quale produrr\`a delle misure giocoforza affette da errori casuali. Si supponga di sostituire $a$ nell'equazione $z(t)$ con una sua perturbazione $\tilde{a} = a + \varepsilon$ dove $\varepsilon$ \`e una variazione casuale della misura data dal \emph{rumore} che caratterizza qualsiasi processo di misura. Si pu\'o supporre $Var(\varepsilon) = 0$ e considerare, ai fini di questa trattazione, $\varepsilon$ come un valore costante; in realt\`a $\varepsilon$ \`e una variable casuale a varianza generalmente non nulla. Si suppongano inoltre $v_0 = z_0 = 0$ per comodit\`a di calcolo:
$$
z(t) = \frac{1}{2} \tilde{a} t^2 = \frac{1}{2}(a + \varepsilon) t^2 = \frac{1}{2} \left(at^2 + \varepsilon t^2\right)
$$
Si nota immediatamente che la variazione della misura $z(t)$ data da $\varepsilon$ aumenta con il quadrato del tempo.\\*
\begin{figure}[h]
	\centering
	\includegraphics[scale=0.5]{img/errormeas}
	\caption{Grafico dell' errore di stima della posizione con $a = 10^0, \varepsilon = 10^{-3}$}
	\label{fig:errormeas}
\end{figure}\newpage
Il sistema dinamico individuato in $(3)$, \`e una tipologia di sistema dinamico caratterizzato da assenza di \emph{rumore di processo}: fatta assunzione di conoscere esattamente il valore di $a$, la doppia integrazione di $(3)$ rispetto al tempo fornisce una descrizione esatta e deterministica della dinamica del sistema: la traiettoria sar\`a \emph{esattamente} quella individuata dalla soluzione.\\*
Alcuni processi tuttavia evolvono in parte stocasticamente per loro natura, e questa natura stocastica insita nel processo viene chiamata \emph{rumore di processo}. Si conclude pertanto che non solamente le misurazioni sono affette da rumore, ma anche il processo evolutivo stesso pu\`o essere affetto da rumore stocastico intrinseco. La conseguenza \`e che la forma esplicita delle equazioni $(1)$ e $(2)$ contiene un termine casuale individuato da una variabile aleatoria.\\*
Uno schema di un processo caratterizzato da \emph{rumore} \`e mostrato in figura \ref{fig:mimo1}.
\begin{figure}[h]
	\centering
	\includegraphics{img/mimo1}
	\caption{Processo caratterizzato da \emph{rumore}}
	\label{fig:mimo1}
\end{figure}
\section{I Filtri di Kalman}
Un Filtro di Kalman, o in inglese \emph{Kalman Filter} (KF), \`e, da un punto di vista statistico, uno \emph{stimatore} dello stato di un sistema dinamico caratterizzato da rumore. L'algoritmo individuato dalla formulazione matematica di un KF \`e un caso particolare di SFA.
\subsection{Premesse statistiche}
Un insieme di $N$ sorgenti di misurazioni viene modellato come un insieme di $N$ variabili casuali.\\*
Siano $X_1,\dots,X_N$ $N$ variabili casuali a valori in un insieme finito non vuoto $\mathbb{X}$, e sia $X,Y$ una qualsiasi coppia presa tra le $N$ variabili casuali.\\*
Si chiama \emph{covarianza} di $X,Y$ la seguente quantit\`a:
$$
cov(X,Y) = \mathbb{E}\{[X-\mathbb{E}(X)][Y-\mathbb{E}(Y)]\}= \sigma_{XY}
$$
Si osservi che :
$$
cov(X,X) =  \mathbb{E}\{[X-\mathbb{E}(X)]X-\mathbb{E}(X)]\} = \mathbb{E}[X-\mathbb{E}(X)]^2 =\sigma^2_{X}
$$
Per un vettore di $N$ variabili aleatorie $X_1,\dots,X_N$, si definisce la matrice di \emph{varianza-covarianza}, o semplicemente matrice di \emph{covarianza}, la seguente matrice quadrata $N$\texttt{x}$N$:
$$
\Sigma = \left(\begin{matrix}
cov(X_1,X_1) && cov(X_1,X_2) && \dots && cov(X_1,X_N) \\
\vdots && \vdots && \vdots \\
cov(X_N,X_1) && cov(X_N, X_2) && \dots && cov(X_N,X_N)
\end{matrix}\right) 
$$
$$
 = \left(\begin{matrix}
\sigma^2_{X_1} && \sigma_{X_1,X_2} && \dots && \sigma_{X_1,X_N} \\
\vdots && \vdots && \vdots \\
\sigma_{X_N,X_1} && \sigma_{X_N, X_2} && \dots && \sigma^2_{X_N}
\end{matrix}\right)
$$
Un vettore di $N$ variabili casuali si dice congiuntamente \emph{gaussiano}, ossia distribuito secondo una distribuzione di probabilit\`a \emph{normale multivariata}, quando qualunque combinazione lineare non banale:
$$
Y = \sum_{i=1}^N \alpha_iX_i\;\;\alpha_i \in \mathbb{R}
$$
Ha distribuzione di probabilit\`a \emph{gaussiana}.
\subsection{Filtro di Kalman Lineare}
I KF sono comunemente basati su sistemi dinamici \emph{lineari} a tempo discreto, tuttavia i fenomeni reali sono raramente lineari. Un modello lineare \`e spesso un'approssimazione di un modello pi\`u complesso.\\*
Nel dominio applicativo in cui si colloca la Tesi, ossia quello del posizionamento ferroviario, occorre utilizzare una generalizzazione al caso non-lineare dei Filtri di Kalman standard: il Filtro di Kalman Esteso (CITARE PAPER QUA). Per ragioni di semplicit\`a, in questo capitolo viene esposto il principio base del funzionamento di un semplice KF llineare.
\subsubsection{Definizione del problema}
Si consideri il seguente sistema dinamico lineare discreto:
$$
\begin{cases}
\mathbf x_k = A \mathbf x_{k-1} + B \mathbf u_k + \mathbf w_k \\
\mathbf y_k = C \mathbf x_k + \mathbf v_k
\end{cases}\;\;\;\;(4)
$$
In cui i vettori $\mathbf w_k$ e $\mathbf v_k$ rappresentano rispettivamente il rumore di processo e il rumore di misura. Si assumono \emph{congiuntamente gaussiani}, indipendenti e con matrici di covarianza $Q,R$ rispettivamente.\\*
Il vettore $\mathbf y_k$ rappresenta il vettore di misurazioni campionate all'istante $k$, mentre il vettore $\mathbf x_k$ rappresenta lo stato del sistema all'istante $k$.\\*
Le matrici $A$ e $B$ descrivono la dinamica del modello e si assumono note a priori, pena l'introduzione di errori sistematici, mentre la matrice $C$ descrive la dinamica del processo di misura. Il vettore $\mathbf u_k$ rappresenta l' informazione data in ingresso al sistema al tempo $k$.\\*Uno schema della $(4)$ ad ogni istante di tempo $k$ \`e riportato in figura \ref{fig:mimo2}.\\*
\begin{figure}[t]
	\centering
	\includegraphics[width=\linewidth]{img/mimo2}
	\caption{Processo e misura caratterizzati da rumore}
	\label{fig:mimo2}
\end{figure}
Dal momento che non \`e possibile individuare una soluzione analitica di $(4)$, il problema da risolvere \`e \emph{stimare} lo stato del sistema $\mathbf{x}_k$ per qualunque istante di tempo $k$.
\subsubsection{Soluzione}
Si definiscono i vettori:
\begin{itemize} 
	\item $\hat{\mathbf{x}}^-_k$ come la stima \emph{a priori} dello stato del sistema all'istante $k$, sulla base della conoscenza del processo all'istante $k-1$;
	\item $\hat{\mathbf{x}}_k$ come la stima \emph{a posteriori} dello stato del sistema all'istante $k$, data dalle misurazioni $\mathbf y_k$ allo stesso istante.
\end{itemize}
Si ha che ciascun $\hat{\mathbf{x}}^-_k$ e ciascun $\hat{\mathbf{x}}_k$ \`e in effetti un vettore di variabili aleatorie.\\*
Siano $\mathbf{e}^-_k = (\mathbf x_ k - \hat{\mathbf{x}}^-_k)$, $\mathbf e_k = (\mathbf x_k - \hat{\mathbf{x}}_k)$ rispettivamente l'\emph{errore} a priori e l'\emph{errore} a posteriori di stima, e siano $P^-_k$ e $P_k$ rispettivamente le matrici di covarianza di $\mathbf{e}^-_k$ e di $\mathbf{e}_k$.\\*
Un KF lineare \`e una quintupla di equazioni:
\begin{enumerate}
\item $
\hat{\mathbf{x}}^-_k = A \hat{\mathbf{x}}^-_{k-1} + B \mathbf u_k
$
\item $ P^-_k = A P^-_{k-1} A^T + Q
$
\item $ L_k = P_k^-C^T(CP_k^-C^T + R)^{-1}
$
\item $ \hat{\mathbf{x}}_k = \hat{\mathbf{x}}^-_k + L_k(\mathbf y_k - C \hat{\mathbf{x}}^-_k)
$
\item $ P_k = (I-L_kC)P^-_k
$
\end{enumerate}
In cui le equazioni $1$ e $2$ vengono dette \emph{equazioni di predizione} e proiettano lo stato e la covarianza
dell'errore di stima a priori all'istante temporale $k-1$, in avanti all'istante $k$; mentre le equazioni $3$, $4$, $5$, vengono dette \emph{equazioni di aggiornamento}:
\begin{itemize}
	\item Viene prima calcolata $L_k$, ossia la \emph{matrice dei guadagni di Kalman};
	\item Le misure $\mathbf y_k$ vengono usate per determinare una \emph{stima a posteriori} dello stato del sistema all'istante $k$;
	\item Infine viene calcolata una stima della covarianza dell'errore a posteriori $P_k$.
\end{itemize}
In figura \ref{fig:kf}, \`e riportato uno schema del funzionamento logico di un KF collegato a un sistema dinamico lineare a tempo discreto.\\*
\begin{figure}[h]
	\centering
	\includegraphics[width=\linewidth]{img/kalman}
	\caption{Schema di un KF lineare}
	\label{fig:kf}
\end{figure}
KF \`e un \emph{filtro ricorsivo}, in quanto la stima $\hat{\mathbf{x}}_k$ dello stato del sistema all'istante $k$ viene determinata combinando le informazioni date dal vettore di misurazioni $\mathbf y_k$ campionate all'istante $k$ con la stima dello stato del sistema all'istante $k-1$\footnote{Supposto noto lo stato iniziale $\mathbf x_0$.}.\\*
KF \`e uno \emph{stimatore ottimo}, dove ottimo significa che minimizza la covarianza dell'errore di stima a posteriori, se tutti i rumori hanno distribuzione normale multivariata. 
\newpage
\subsection{Esempio applicativo}
Si supponga di voler determinare posizione e velocit\`a nel sistema dinamico individuato dalla $(3)$. Valgono le seguenti ipotesi:
\begin{itemize}
	\item Il corpo, modellato come puntiforme, viene lasciato cadere da una quota $z(0) = z_0$ assegnata, con velocit\`a iniziale $v_0$;
	\item La sola accelerazione a cui \`e soggetto il corpo \`e l'accelerazione gravitazionale $-g$;
	\item Il sistema non \`e caratterizzato da rumore di processo;
	\item Un osservatore \`e in grado di misurare la quota dell'oggetto mediante uno strumento di misura $X$ 
	distribuito come una normale univariata con varianza $R$;
	\item Il sistema viene osservato ogni secondo per un intervallo di tempo lungo $t_{M}\;s$
\end{itemize}
Valori numerici:
\begin{itemize}
	\item $z_0 = 100\;m$
	\item $v_0 = 0 \frac{m}{s}$
	\item $g = 1\;\frac{m}{s^2}$
	\item $R = 1\;m^2$
	\item $t_M = 10\;s$
\end{itemize}
Si scrivono le equazioni $(4)$ modellanti la dinamica del processo evolutivo e la dinamica del processo di misurazione:
$$
\begin{cases}
\mathbf x_k = A \mathbf x_{k-1} + B \mathbf u_k + \mathbf w_k \\
\mathbf y_k = C \mathbf x_k + \mathbf v_k
\end{cases} = 
\begin{cases}
\mathbf x_k = \left(\begin{matrix}
1 & 1 \\
0 & 1 
\end{matrix}\right) \mathbf x_{k-1} + \left(\begin{matrix}
\frac{1}{2} \\
1
\end{matrix}\right) -g + \underline 0 \\
\mathbf y_k = \left(\begin{matrix}
1\\
0
\end{matrix}\right) \mathbf x_k + \mathbf v_k
\end{cases} = 
$$
$$
= \begin{cases}
\mathbf x_k = \left(\begin{matrix}
1 & 1 \\
0 & 1 
\end{matrix}\right) \mathbf x_{k-1} - \left(\begin{matrix}
\frac{1}{2} \\
1
\end{matrix}\right) \\
\mathbf y_k = \left(\begin{matrix}
1\\
0
\end{matrix}\right) \mathbf x_k + \mathbf v_k
\end{cases}
$$
Il Filtro di Kalman \`e dato dalle seguenti equazioni:
\begin{itemize}
	\item \begin{enumerate}
		\item[$1.$] $
		\hat{\mathbf{x}}^-_k = A \hat{\mathbf{x}}^-_{k-1} + B \mathbf u_k =$\\*
		$= \left(\begin{matrix}
		1 & 1 \\
		0 & 1 
		\end{matrix}\right) \hat{\mathbf{x}}^-_{k-1} - \left( \begin{matrix}
		\frac{1}{2}\\
		1
		\end{matrix}\right)
		$
		\item[$2.$]  $ P^-_k = A P^-_{k-1} A^T + Q =$\footnote{Non essendoci rumore di processo, la matrice $Q$ di covarianza di tale quantit\`a \`e la matrice nulla.}\\*
		$= \left(\begin{matrix}
		1 & 1 \\
		0 & 1 
		\end{matrix}\right) P^-_{k-1}\left(\begin{matrix}
		1 & 1 \\
		0 & 1 
		\end{matrix}\right)^T + 0$        
	\end{enumerate}
	\item
	\begin{enumerate}
		\item[$3.$] $ L_k = P_k^-C^T(CP_k^-C^T + R)^{-1}
		 = $\\*
		 $ = P_k^-\left(\begin{matrix}
		 1\\
		 0
		 \end{matrix}\right)^T \left[ \left(\begin{matrix}
		 1\\
		 0
		 \end{matrix}\right)P_k^-\left(\begin{matrix}
		 1\\
		 0
		 \end{matrix}\right)^T + 1
		 \right]^{-1} =
		 $
		\item[$4.$] $ \hat{\mathbf{x}}_k = \hat{\mathbf{x}}^-_k + L_k(\mathbf y_k - C \hat{\mathbf{x}}^-_k) =
		$\\*
		$
		= \hat{\mathbf{x}}^-_k + L_k\left[ \mathbf y_k - \left(\begin{matrix}
		1\\
		0
		\end{matrix}\right) \hat{\mathbf{x}}^-_k\right]
		$
		\item[$5.$] $ P_k = (I-L_kC)P^-_k =
		$
		\\*
		$
		= \left[ \left(\begin{matrix}
		1 & 0\\
		0 & 1
		\end{matrix}\right)-L_k\left(\begin{matrix}
		1\\
		0
		\end{matrix}\right)\right]P^-_k
		$
	\end{enumerate}
\end{itemize}
Assegnato lo stato iniziale:
$$
\mathbf x_0 = (z_0,v_0) = (100, 0)
$$
Si ha che la stima a priori dello stato del sistema all'istante $0$ vale esattamente lo stato iniziale noto del sistema:
$$
\hat{\mathbf{x}}^-_{0} = \mathbf x_0 = (z_0,v_0) = (100,0)
$$
Mentre la matrice di covarianza dell'errore a priori viene inizializzata come la matrice di covarianza della sorgente delle misurazioni:
$$
P^-_{0} = R = 1
$$
Supponendo di effettuare, attraverso lo strumento $X$, le misure riportate in tabella \ref{tab:misurekalman}, un esempio di implementazione in linguaggio MATLAB\footnote{Le motivazioni della scelta di MATLAB sono da ricercarsi nella natura del linguaggio: esso \`e fortemente orientato al calcolo numerico e alla manipolazione efficiente di espressioni matriciali.} del KF descritto \`e mostrato di seguito.
\begin{table}[h]
	\centering
	\begin{tabular}{|c|c|c|c|c|c|c|c|c|c|c|}
		\hline 
		$\mathbf{t\;(s)}$ & $1$ & $2$ & $3$ & $4$ & $5$ & $6$ & $7$ & $8$ & $9$ & $10$ \\ 
		\hline 
		$\mathbf{y_t\;(m)}$ & $100$ & $97.9$ & $94.9$ & $92.7$ & $87.3$ & $81.3$ & $75.8$ & $67.5$ & $59.17$ &$51.1$ \\ 
		\hline 
	\end{tabular} 
	\caption{Misurazioni di esempio del corpo in caduta libera}
	\label{tab:misurekalman}
\end{table}
\subsubsection{Codice Soluzione}
Si riporta in questa sezione il codice risolutivo del problema individuato.
\lstinputlisting[language=MATLAB, firstline=1, lastline=7, caption ={Definizione delle variabili del problema}]{Models/kalmanexample.m}

\lstinputlisting[language=MATLAB, firstline=11, lastline=20, caption ={Inizializzazione del Filtro di Kalman}]{Models/kalmanexample.m}
\newpage
\lstinputlisting[language=MATLAB, firstline=22, lastline=38, caption ={Algoritmo individuato dalle Equazioni di KF}]{Models/kalmanexample.m}
\newpage
\subsubsection{Risultati}
Nelle figure \ref{fig:predspeed}, \ref{fig:predheight} viene mostrato il grafico che compara i valori stimati di velocit\`a e posizione con i veri valori delle medesime grandezze; infatti tali valori sono determinabili analiticamente come soluzioni esplicite della $(3)$.\\*
Tali grafici danno un'idea immediata della capacit\`a che ha l'algoritmo di stimare la posizione e la velocit\`a in maniera affidabile anche in presenza di un processo di misura affetto da rumore.\\*
\begin{figure}[h]
	\centering
	\includegraphics[scale=0.7]{img/predspeed}
	\caption{Stima della velocit\`a del corpo in caduta}
	\label{fig:predspeed}
\end{figure}
\begin{figure}[h]
	\centering
	\includegraphics[scale=0.7]{img/predheight}
	\caption{Stima della posizione del corpo in caduta}
	\label{fig:predheight}
\end{figure}
\newpage
In tabella \ref{tab:syntex} viene riportata una sintesi dei dati prodotti dall'esecuzione dell'algoritmo.\\*
\begin{table}[h]
	\begin{tabular}{|c|c|c|c|c|c|}
		\hline 
		$\mathbf{t}\;(s)$ & \textbf{Misura (m)} & \textbf{Pos.(m)} & \textbf{Stima Pos. (m)} & \textbf{Vel.} $\mathbf{\left(\frac{m}{s}\right)}$ & \textbf{Stima Vel.} $\mathbf{\left(\frac{m}{s}\right)}$ \\ 
		\hline 
		$1$ & $100$ & $99.5$ & $99.833$ & $-1$ & $-0.83333$ \\ 
		\hline 
		$2$ & $97.9$ & $98$ & $97.9167$ & $-2$ & $-2.0333$ \\ 
		\hline 
		$3$ & $94.9$ & $95.5$ & $94.9545$ & $-3$ & $-3.1636$ \\ 
		\hline 
		$4$ & $92.7$ & $92$ & $92.6611$ & $-4$ & $-3.8444$ \\ 
		\hline 
		$5$ & $83.7$ & $87.5$ & $87.3074$ & $-5$ & $-5.037$ \\ 
		\hline 
		$6$ & $81.3$ & $82$ & $81.3184$ & $-6$ & $-6.1105$ \\ 
		\hline 
		$7$ & $75.8$ & $75.5$ & $75.7941$ & $-7$ & $-6.9588$ \\ 
		\hline 
		$8$ & $67.5$ & $68$ & $67.5076$ & $-8$ & $-8.0606$ \\ 
		\hline 
		$9$ & $59.17$ & $59.5$ & $59.174$ & $-9$ & $-9.0358$ \\ 
		\hline 
		$10$ & $51.1$ & $50$ & $51.0892$ & $-10$ & $-9.8922$ \\ 
		\hline 
	\end{tabular} 
	\caption{Risultati generali dell'algoritmo}
	\label{tab:syntex}
\end{table}
Nelle tabelle \ref{tab:errors} e \ref{tab:errorsmeas} \`e mostrato un paragone fra l'errore a posteriori $(\mathbf x_k - \mathbf{\tilde{x}}_k)$ e l'errore della misura rispetto alla posizione reale.\\*
\begin{table}[h]
	\begin{tabular}{|c|c|c|c|}
		\hline 
		$\mathbf{t}\;(s)$ & $\mathbf{\hat{x}}_k$ & $\mathbf x_k$ & $\mathbf x_k - \mathbf{\hat{x}}_k$ \\ 
		\hline 
		$1$&$(99.833,-0.83333)$&$(99.5,-1)$  & $(-0.333333,-0.166667)$
		   \\ 
		\hline 
		$2$&$(97.9167,-2.0333)$ & $(98,-2)$ & $(0.083333,0.033333)$
		
		   \\ 
		\hline 
		$3$& $(94.9545,-3.1636)$ & $(95.5,-3)$ & $(0.545455,0.163636)$
		
		  \\ 
		\hline 
		$4$&  $(92.6611,-3.8444)$ & $(92,-4)$ &$(-0.661111,-0.155556)$
		
		  \\ 
		\hline 
		$5$&$(87.3074,-5.037)$  & $(87.5,-5)$ &  $(0.192593,0.037037)$
		  \\ 
		\hline 
		$6$& $(81.3184,-6.1105)$ & $(82,-6)$ &  $(0.681579,0.110526)$
		  \\ 
		\hline 
		$7$&  $(75.7941,-6.9588)$ & $(75.5,-7)$ &  $(-0.294118,-0.041176)$
		  \\ 
		\hline 
		$8$& $(67.5076,-8.0606)$ & $(68,-8)$ &  $(0.492424,0.060606)$
		  \\ 
		\hline 
		$9$&  $(59.174,-9.0358)$ & $(59.5,-9)$ &  $(0.326024,0.035783)$
		  \\ 
		\hline 
		$10$& $(51.0892,-9.8922)$ & $(50,-10)$ & $(-1.089216,-0.107843)$
		  \\ 
		\hline 
	\end{tabular} 
	\caption{Errori a posteriori}
	\label{tab:errors}
\end{table}
\begin{table}[h]
	\begin{tabular}{|c|l|r|}
		\hline 
		$\mathbf{t}\;(s)$ & \textbf{Errore a posteriori (pos.) (m)} & $\Delta \mathbf{(pos,misura)}$ \textbf{(m)} \\ 
		\hline 
		$1$ & $-0.333333$ & $(99.500 - 100) =-0.5$ \\ 
		\hline 
		$2$ & $0.083333$ & $(98 - 97.9) = 0.1$ \\ 
		\hline 
		$3$ & $0.545455$ & $(95.5-94.9) = 0.6$ \\ 
		\hline 
		$4$ & $-0.661111$ & $(92 - 92.7) = -0.7$ \\ 
		\hline 
		$5$ &  $0.192593$ & $(87.5 - 87.3) = 0.2$ \\ 
		\hline 
		$6$ & $0.681579$ & $(82 - 81.3) = 0.7$ \\ 
		\hline 
		$7$ & $-0.294118$ & $(75.5 - 75.8) = -0.3$ \\ 
		\hline 
		$8$ & $0.492424$ & $(68 - 67.5) = 0.5$ \\ 
		\hline 
		$9$ & $0.326024$ & $(59.5 - 59.17) = 0.33$ \\ 
		\hline 
		$10$ & $-1.089216$ & $(50-51.1) = -1.1$ \\ 
		\hline 
	\end{tabular} 
	\caption{Confronto tra errore a posteriori sulla posizione, e distanza fra valori reali di posizione e misure di posizione}
	\label{tab:errorsmeas}
\end{table}
Si noti che per ciascun istante $t$, l'errore a posteriori sul valore di posizione \`e minore, in valore assoluto, della distanza fra la misura osservata e la posizione reale del corpo.\\*
Questo porta alla conclusione che l'utilizzo di KF ha permesso di ottenere misure sempre pi\`u affidabili rispetto a quelle ottenute dalla sorgente.\\*
\chapter{Architettura di Sistema}
Il sistema studiato si compone dei seguenti \emph{Constituent Systems}:
\begin{itemize}
	\item \emph{Sensor Set}, ossia un insieme di sensori atto a campionare le misure di interesse per il sistema. Il \emph{Sensor Set} \`e composto dai seguenti strumenti di misura:
	\begin{itemize}
		\item \emph{Inertial Measurment Unit} (IMU), ovvero un circuito che integra un accelerometro, un giroscopio e un magnetometro. IMU campiona e fornisce al sistema le misure di accelerazione, velocit\`a angolare e campo magnetico lungo i tre assi.
		\item Odometro, ovvero un sistema composto da un emettitore e da un ricevitore radar, che fornisce al sistema le misure di velocit\`a lineare del treno attraverso il tempo impiegato da una ruota a compiere un giro completo.
		\item Ricevitore GPS, ovvero un hardware in grado di ricevere informazioni sulle coordinate geografiche del treno. Fornisce le misure di posizione al sistema.

	\end{itemize}
	\item Piattaforma di elaborazione dati. Consiste di una scheda \texttt{Nvidia TX-Jetson} su cui viene eseguito l'algoritmo di posizionamento \emph{Sensor Fusion}.
	\item \emph{On Board Control Unit} (OBCU). Computer di bordo del treno, riceve le informazioni di posizione del treno dalla piattaforma di elaborazione dati, e comunica con il sistema di \emph{interlocking} della traccia.
\end{itemize}
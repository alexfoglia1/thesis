\chapter{Architettura di Sistema}
In questo capitolo viene descritta l'architettura hardware del sottosistema di posizionamento, in particolare, ne vengono evidenziati i moduli costituenti e le loro interfacce di comunicazione. Vengono infine descritte le interazioni osservabili.
\section{Descrizione generale}
Lo scopo del sistema \`e quello di implementare un meccanismo di posizionamento basato su SFA.\\*
Tale algoritmo viene eseguito da una libreria software schematizzabile, ai fini di questa Tesi, come una \emph{black-box} che rappresenta il nucleo centrale del sottosistema di posizionamento.\\*
Ricevuti in ingresso un certo insieme di misure, essa fornisce in uscita una stima della posizione del treno, pi\`u accurata della stima che si otterrebbe utilizzando le misure provenienti dai singoli sensori.\cite{datafuse} \\*
\begin{figure}[h]
	\centering
	\includegraphics[scale=0.6]{img/sfaschema}
	\caption{Schema SFA}
	\label{fig:sfa}
\end{figure}
\clearpage
SFA viene eseguito su di un hardware installato a bordo treno, e la sua esecuzione \`e volta a monitorare costantemente il moto del treno.\\*
Il ciclo di esecuzione di SFA \`e essenzialmente una continua iterazione di due distinte fasi logiche:
\begin{itemize}
	\item Acquisizione delle misure;
	\item Predizione della posizione del treno.
\end{itemize}
Le grandezze fisiche che dovranno essere misurate e fornite a SFA sono:
\begin{itemize}
	\item Vettore accelerazione;
	\item Vettore velocit\`a angolare;
	\item Coordinate geografiche;
	\item Velocit\`a lineare (scalare).
\end{itemize}
In quest'applicazione, SFA utilizza tali informazioni in combinazione con un'apposita digitalizzazione della traccia tramviaria su cui si trova il treno monitorato.\cite{sfaimugps}\cite{sfaimuodo}\cite{sfaimuodogps} \\* Queste informazioni si suppongono note a priori ed accedibili tramite un \emph{database} caricato in memoria centrale. \cite{sqlite3}
\section{Sistemi Costituenti}
Il sottosistema di posizionamento si compone dei moduli, o sistemi costituenti, descritti nel seguito di questa sezione.
\subsection{Sensor Set}
Il \emph{Sensor Set} \`e un insieme di sensori atti a campionare le misure richieste da SFA. Esso si compone a sua volta dei seguenti moduli:
\begin{itemize}
	\item \emph{Inertial Measurement Unit} (IMU):\\*
		Sensore inerziale incaricato di campionare e trasmettere a SFA i vettori \texttt{accelerazione} ($\mathbf{a}$) e \texttt{velocit\`a angolare} ($\mathbf{v_{ang}}$). Le misure di IMU sono prese rispetto alla Terra e sono espresse in unit\`a stabilite dallo standard internazionale (SI):
		$$
		\mathbf{a}\;\left[\frac{m}{s^2}\right]\;\;\;\;\mathbf{v_{ang}}\;\left[ \frac{rad}{s} \right]
		$$
		IMU \`e il sensore principale su cui si basa SFA nel predire la posizione del treno. Date le caratteristiche intrinseche del particolare SFA utilizzato, ossia un \emph{Filtro di Kalman}, il sistema funziona anche senza i rimanenti sensori. Si osserverebbe tuttavia un calo delle performance in termini di errore commesso sulla stima della posizione del treno. \cite{partialmeas} \cite{gpsdarkarea}
		\begin{figure}[h]
			\centering
			\includegraphics[scale=0.5]{img/imu}
			\caption{\emph{Inertial Measurment Unit}}
			\label{fig:imu}
		\end{figure}
		\item Odometro:\\*
		Unit\`a incaricata di fornire a SFA i valori di velocit\`a lineare del treno, espressi in $\frac{m}{s}$.
		\item GPS:
		\\*Unit\`a che fornisce a SFA le misure di posizione del treno.\\*
Le misure di GPS sono riportate in formato standard come tripla di coordinate \texttt{(latitudine, longitudine, altitudine)}, rispettivamente espresse in gradi \texttt{N-S}, in gradi \texttt{E-O} e in \texttt{metri} sul livello del mare.
\begin{figure}[h]
	\centering
	\includegraphics[scale=0.3]{img/gpsublox}
	\caption{Ricevitore \texttt{GPS ublox EVK-M8T}}
	\label{fig:gpsublox}
\end{figure}
\end{itemize}
\subsection{Piattaforma di elaborazione dati}
La piattaforma di elaborazione dati \`e l'hardware sul quale viene eseguito SFA.
Consiste di una scheda \texttt{Nvidia TX-Jetson} collegata al \emph{Sensor Set}.\\*
Da quest'ultimo essa riceve le misure da processare tramite SFA.
\begin{figure}[h]
		\centering
		\includegraphics[width=0.7\linewidth]{img/nvidia}
		\caption{\texttt{Nvidia TX-Jetson}}
		\label{fig:nvidia}
\end{figure}
\texttt{Nvidia TX-Jetson} \`e un'architettura specifica per sistemi \emph{embedded}. Essa \`e ottimizzata per i calcoli computazionalmente onerosi tipici delle applicazioni di intelligenza artificiale.\cite{nvidia}\\*
\begin{table}[h]
\begin{tabular}{|p{3cm}|p{8cm}|}
	\hline 
	\textbf{GPU} & 256-core NVIDIA Pascal\\ 
	\hline 
	\textbf{CPU} & Dual-Core NVIDIA Denver 2 64-Bit CPU Quad-Core ARM Cortex-A57 MPCore \\ 
	\hline 
	\textbf{Memoria} & 8GB 128-bit LPDDR4 Memory \\ 
	\hline 
	\textbf{Storage} & 32GB eMMC 5.1 \\ 
	\hline 
	\textbf{Alimentazione} & 7.5W / 15W \\ 
	\hline 
\end{tabular}
\caption{Specifiche Tecniche \texttt{NVidia TX-Jetson}}
\label{tab:nvidia}
\end{table}
\subsection{On Board Control Unit}
L'\emph{On Board Control Unit} (OBCU) \`e il computer di bordo del treno. Esso non svolge alcun ruolo attivo nel sistema di posizionamento, tuttavia la progressiva chilometrica, stimata da SFA, dovr\'a essere trasmessa a OBCU al fine di poter utilizzare questa informazione all'interno del sistema di \emph{interlocking} della traccia.
	\section{Specifica delle Interfacce}
	\subsection{Relied Upon Interfaces}
	Le interfacce sono definite come punti di interazione, tra un CS e l'ambiente oppure tra un CS e un altro.\\*
	In questa sezione si evidenziano le principali interfacce del sistema, alle quali si osservano le interazioni fondamentali che avvengono al suo interno.\\*
	Tali interfacce prendono il nome di \emph{Relied Upon Interfaces} (RUI). Le RUI si dividono in:
	\begin{itemize}
		\item \emph{Relied Upon Physical Interfaces} (RUPI), in cui l'interazione avviene tramite osservazione diretta di una grandezza fisica;
		\item \emph{Relied Upon Message Interfaces} (RUMI), dove l'interazione \`e rappresentata da uno scambio di messaggi a livello \emph{cyber}.
	\end{itemize}
	La specifica delle RUI \`e di particolare importanza poich\'e qualunque struttura del sistema, responsabile del comportamento osservato, pu\'o essere ridotta alla specifica delle interfacce del sistema. \cite{interfacespec}\\*  
	Il CPS interagisce con l'ambiente attraverso le RUPI del \emph{Sensor Set}, ossia gli strumenti di misura che esso integra. Queste interfacce acquisiscono, a diverse frequenze, i dati sul moto del treno che verranno elaborati da SFA. Una descrizione sintetica delle RUPI del sistema \`e mostrata in tabella \ref{tab:rupi}.\\*
	\begin{table}[h]
	\centering
	\begin{tabular}{|c|c|c|}
		\hline 
		\textbf{RUPI} & \textbf{Grandezza Campionata}  & \textbf{Parti interagenti} \\ 
		\hline 
		Accelerometro & Accelerazione & Ambiente - IMU \\ 
		\hline 
		Giroscopio & Velocit\`a angolare & Ambiente - IMU  \\ 
		\hline 
		Radar & Velocit\`a lineare & Ambiente - Odometro \\ 
		\hline 
		Ricevitore GPS & Coordinate geografiche& Ambiente - GPS \\ 
		\hline 
	\end{tabular}
	\caption{Specifica delle RUPI del sistema}
	\label{tab:rupi}
	\end{table}\clearpage
	Per quanto concerne le RUMI, se ne osservano di due tipi:
	\begin{itemize}
		\item Tre bus dati, che collegano il \emph{Sensor Set} alla scheda \texttt{Nvidia TX-Jetson}. Su ciascuno di essi, \emph{Sensor Set} invia rispettivamente messaggi contenenti i dati campionati da IMU, Odometro e GPS.
		\item Interfaccia LTE. Essa permette di realizzare una \emph{rete wireless ad hoc} fra la scheda e OBCU.\\*
		All'interno di tale rete vengono instradati datagrammi \texttt{IP} contenenti le informazioni sulla progressiva chilometrica stimata da SFA, ed eventualmente messaggi di \emph{acknowledgment} di OBCU verso la scheda.
		\begin{figure}[h]
			\centering
			\includegraphics[scale=0.40]{img/lte}
			\caption{Modem \texttt{TP-LINK M7350 LTE-4G}}
			\label{fig:lte}
		\end{figure}
	\end{itemize}
	\begin{figure}[h]
		\centering
		\includegraphics[width=0.7\linewidth]{img/TrainDiagram}
		\caption{Architettura hardware del sottosistema di posizionamento}
		\label{fig:tdiagram}
	\end{figure}
		\begin{table}[h]
		\centering
		\begin{tabular}{|c|c|c|}
			\hline 
			\textbf{RUMI} & \textbf{Informazione trasmessa}  & \textbf{Parti interagenti} \\ 
			\hline 
			Bus Dati 1 & Accelerazione, Velocit\`a angolare & Sensor Set - \texttt{NVidia TX-Jetson} \\ 
			\hline 
			Bus Dati 2 & Velocit\`a lineare & Sensor Set - \texttt{NVidia TX-Jetson} \\ 
			\hline 
			Bus Dati 3 & Coordinate geografiche & Sensor Set - \texttt{NVidia TX-Jetson} \\ 
			\hline 
			LTE & Posizione del treno & \texttt{NVidia TX-Jetson} - OBCU \\ 
			\hline 
		\end{tabular}
		\caption{Specifica delle RUMI del sistema}
		\label{tab:rumi}
	\end{table}
\newpage
	\section{Interazioni}
	In questa sezione si descrivono le interazioni osservabili alle interfacce sopra descritte. Queste possono essere in prima istanza categorizzate in accordo a discrezione della fase di SFA in cui esse avvengono.\\*
	Si distinguono pertanto le interazioni riguardanti l'acquisizione dei dati in ingresso a SFA, e le interazioni riguardanti l'acquisizione da parte di OBCU della posizione del treno.
	\subsection{Acquisizione dei dati}
	L'acquisizione dei dati si divide in due differenti interazioni: la prima, con l'ambiente, avviene alle RUPI del \emph{Sensor Set}, mentre la seconda avviene alle RUMI bus dati che collegano il \emph{Sensor Set} alla piattaforma di elaborazione dati.\\*
	I moduli che compongono il \emph{Sensor Set} campionano ad una data frequenza le grandezze fisiche che descrivono il moto del treno. Ciascun campionamento fisico \`e seguito dall'invio dei valori letti alla piattaforma di elaborazione dati. I moduli del \emph{Sensor Set} sono tra di loro indipendenti.\\* 
	In figura \ref{fig:seqdiag} viene riportato un \emph{sequence diagram} rappresentante una possibile sequenza di campionamento e invio dei dati.\newpage	\begin{figure}[h]
		\centering
		\includegraphics[width=0.7\linewidth]{img/seqdiag}
		\caption{Sequenza di acquisizione dati}
		\label{fig:seqdiag}
	\end{figure}
	Questa tipologia di interazione \`e detta \emph{time-triggered}, in quanto \`e determinata unicamente dallo scorrere del tempo. \cite{timetriggered}
	\subsection{Trasmissione della posizione}
	La piattaforma di elaborazione dati esegue SFA durante l'intero moto del treno. Le misure fornite dai sensori vengono elaborate al fine di aggiornare continuamente la stima della posizione del treno.\\*
	Ogniqualvolta un'aggiornamento di SFA viene completato, avviene un'interazione all'interfaccia LTE. Tale interazione consiste nell'invio di un messaggio contenente la posizione del treno, dalla piattaforma di elaborazione dati verso OBCU, e nella trasmissione di un messaggio di \emph{acknowledgment} nel senso opposto.\\*
	La tipologia di scambio dei messaggi esposta \`e detta \emph{event-triggered} \cite{evttimetriggered} in quanto le tempistiche di interazione non sono note a priori, ma dipendono dal tempo impiegato da SFA a compiere un'iterazione per aggiornare la stima prodotta.\\*
	LTE \`e a tutti gli effetti una regolare interfaccia di rete. Il messaggio trasmesso \`e contenuto nel \emph{payload} di un datagramma \texttt{UDP}; in accordo al modello di rete \texttt{ISO-OSI}. \cite{libroreti}
	\begin{figure}[h]
		\centering
		\includegraphics[width=0.7\linewidth]{img/seqdiag2}
		\caption{Sequenza di trasmissione della posizione}
		\label{fig:seqdiag2}
	\end{figure}